%%
%% Beginning of file 'sample62.tex'
%%
%% Modified 2018 January
%%
%% This is a sample manuscript marked up using the
%% AASTeX v6.2 LaTeX 2e macros.
%%
%% AASTeX is now based on Alexey Vikhlinin's emulateapj.cls 
%% (Copyright 2000-2015).  See the classfile for details.

%% AASTeX requires revtex4-1.cls (http://publish.aps.org/revtex4/) and
%% other external packages (latexsym, graphicx, amssymb, longtable, and epsf).
%% All of these external packages should already be present in the modern TeX 
%% distributions.  If not they can also be obtained at www.ctan.org.

%% The first piece of markup in an AASTeX v6.x document is the \documentclass
%% command. LaTeX will ignore any data that comes before this command. The 
%% documentclass can take an optional argument to modify the output style.
%% The command below calls the preprint style  which will produce a tightly 
%% typeset, one-column, single-spaced document.  It is the default and thus
%% does not need to be explicitly stated.
%%
%%
%% using aastex version 6.2
\documentclass{aastex62}

%% The default is a single spaced, 10 point font, single spaced article.
%% There are 5 other style options available via an optional argument. They
%% can be envoked like this:
%%
%% \documentclass[argument]{aastex62}
%% 
%% where the layout options are:
%%
%%  twocolumn   : two text columns, 10 point font, single spaced article.
%%                This is the most compact and represent the final published
%%                derived PDF copy of the accepted manuscript from the publisher
%%  manuscript  : one text column, 12 point font, double spaced article.
%%  preprint    : one text column, 12 point font, single spaced article.  
%%  preprint2   : two text columns, 12 point font, single spaced article.
%%  modern      : a stylish, single text column, 12 point font, article with
%% 		  wider left and right margins. This uses the Daniel
%% 		  Foreman-Mackey and David Hogg design.
%%  RNAAS       : Preferred style for Research Notes which are by design 
%%                lacking an abstract and brief. DO NOT use \begin{abstract}
%%                and \end{abstract} with this style.
%%
%% Note that you can submit to the AAS Journals in any of these 6 styles.
%%
%% There are other optional arguments one can envoke to allow other stylistic
%% actions. The available options are:
%%
%%  astrosymb    : Loads Astrosymb font and define \astrocommands. 
%%  tighten      : Makes baselineskip slightly smaller, only works with 
%%                 the twocolumn substyle.
%%  times        : uses times font instead of the default
%%  linenumbers  : turn on lineno package.
%%  trackchanges : required to see the revision mark up and print its output
%%  longauthor   : Do not use the more compressed footnote style (default) for 
%%                 the author/collaboration/affiliations. Instead print all
%%                 affiliation information after each name. Creates a much
%%                 long author list but may be desirable for short author papers
%%
%% these can be used in any combination, e.g.
%%
%% \documentclass[twocolumn,linenumbers,trackchanges]{aastex62}
%%
%% AASTeX v6.* now includes \hyperref support. While we have built in specific
%% defaults into the classfile you can manually override them with the
%% \hypersetup command. For example,
%%
%%\hypersetup{linkcolor=red,citecolor=green,filecolor=cyan,urlcolor=magenta}
%%
%% will change the color of the internal links to red, the links to the
%% bibliography to green, the file links to cyan, and the external links to
%% magenta. Additional information on \hyperref options can be found here:
%% https://www.tug.org/applications/hyperref/manual.html#x1-40003
%%
%% If you want to create your own macros, you can do so
%% using \newcommand. Your macros should appear before
%% the \begin{document} command.
%%
\newcommand{\vdag}{(v)^\dagger}
\newcommand\aastex{AAS\TeX}
\newcommand\latex{La\TeX}

%% Tells LaTeX to search for image files in the 
%% current directory as well as in the figures/ folder.
\graphicspath{{./}{figures/}}

%% Reintroduced the \received and \accepted commands from AASTeX v5.2
\received{XX XX, 2019}
\revised{XX XX, 2019}
\accepted{XX XX, 2019}
%% Command to document which AAS Journal the manuscript was submitted to.
%% Adds "Submitted to " the arguement.
\submitjournal{ApJ}

%% Mark up commands to limit the number of authors on the front page.
%% Note that in AASTeX v6.2 a \collaboration call (see below) counts as
%% an author in this case.
%
%\AuthorCollaborationLimit=3
%
%% Will only show Schwarz, Muench and "the AAS Journals Data Scientist 
%% collaboration" on the front page of this example manuscript.
%%
%% Note that all of the author will be shown in the published article.
%% This feature is meant to be used prior to acceptance to make the
%% front end of a long author article more manageable. Please do not use
%% this functionality for manuscripts with less than 20 authors. Conversely,
%% please do use this when the number of authors exceeds 40.
%%
%% Use \allauthors at the manuscript end to show the full author list.
%% This command should only be used with \AuthorCollaborationLimit is used.

%% The following command can be used to set the latex table counters.  It
%% is needed in this document because it uses a mix of latex tabular and
%% AASTeX deluxetables.  In general it should not be needed.
%\setcounter{table}{1}

%%%%%%%%%%%%%%%%%%%%%%%%%%%%%%%%%%%%%%%%%%%%%%%%%%%%%%%%%%%%%%%%%%%%%%%%%%%%%%%%
%%
%% The following section outlines numerous optional output that
%% can be displayed in the front matter or as running meta-data.
%%
%% If you wish, you may supply running head information, although
%% this information may be modified by the editorial offices.
\shorttitle{Seed paper}
\shortauthors{Li \& Morton et al.}
%%
%% You can add a light gray and diagonal water-mark to the first page 
%% with this command:
% \watermark{text}
%% where "text", e.g. DRAFT, is the text to appear.  If the text is 
%% long you can control the water-mark size with:
%  \setwatermarkfontsize{dimension}
%% where dimension is any recognized LaTeX dimension, e.g. pt, in, etc.
%%
%%%%%%%%%%%%%%%%%%%%%%%%%%%%%%%%%%%%%%%%%%%%%%%%%%%%%%%%%%%%%%%%%%%%%%%%%%%%%%%%

%% This is the end of the preamble.  Indicate the beginning of the
%% manuscript itself with \begin{document}.

\begin{document}

\title{Constrain rotational period of KOI stellar using Gaussian Process}

%% LaTeX will automatically break titles if they run longer than
%% one line. However, you may use \\ to force a line break if
%% you desire. In v6.2 you can include a footnote in the title.

%% A significant change from earlier AASTEX versions is in the structure for 
%% calling author and affilations. The change was necessary to implement 
%% autoindexing of affilations which prior was a manual process that could 
%% easily be tedious in large author manuscripts.
%%
%% The \author command is the same as before except it now takes an optional
%% arguement which is the 16 digit ORCID. The syntax is:
%% \author[xxxx-xxxx-xxxx-xxxx]{Author Name}
%%
%% This will hyperlink the author name to the author's ORCID page. Note that
%% during compilation, LaTeX will do some limited checking of the format of
%% the ID to make sure it is valid.
%%
%% Use \affiliation for affiliation information. The old \affil is now aliased
%% to \affiliation. AASTeX v6.2 will automatically index these in the header.
%% When a duplicate is found its index will be the same as its previous entry.
%%
%% Note that \altaffilmark and \altaffiltext have been removed and thus 
%% can not be used to document secondary affiliations. If they are used latex
%% will issue a specific error message and quit. Please use multiple 
%% \affiliation calls for to document more than one affiliation.
%%
%% The new \altaffiliation can be used to indicate some secondary information
%% such as fellowships. This command produces a non-numeric footnote that is
%% set away from the numeric \affiliation footnotes.  NOTE that if an
%% \altaffiliation command is used it must come BEFORE the \affiliation call,
%% right after the \author command, in order to place the footnotes in
%% the proper location.
%%
%% Use \email to set provide email addresses. Each \email will appear on its
%% own line so you can put multiple email address in one \email call. A new
%% \correspondingauthor command is available in V6.2 to identify the
%% corresponding author of the manuscript. It is the author's responsibility
%% to make sure this name is also in the author list.
%%
%% While authors can be grouped inside the same \author and \affiliation
%% commands it is better to have a single author for each. This allows for
%% one to exploit all the new benefits and should make book-keeping easier.
%%
%% If done correctly the peer review system will be able to
%% automatically put the author and affiliation information from the manuscript
%% and save the corresponding author the trouble of entering it by hand.

%\correspondingauthor{August Muench}
%\email{greg.schwarz@aas.org, gus.muench@aas.org}

\author{Yangyang Li}
\affil{University of Florida, Gainesville, FL}

\author{Timothy Morton}
\affiliation{University of Florida, Gainesville, FL}

%% Note that the \and command from previous versions of AASTeX is now
%% depreciated in this version as it is no longer necessary. AASTeX 
%% automatically takes care of all commas and "and"s between authors names.

%% AASTeX 6.2 has the new \collaboration and \nocollaboration commands to
%% provide the collaboration status of a group of authors. These commands 
%% can be used either before or after the list of corresponding authors. The
%% argument for \collaboration is the collaboration identifier. Authors are
%% encouraged to surround collaboration identifiers with ()s. The 
%% \nocollaboration command takes no argument and exists to indicate that
%% the nearby authors are not part of surrounding collaborations.

%% Mark off the abstract in the ``abstract'' environment. 
\begin{abstract}

Based on Morton \& Winn 2014 et al., we can obtain obliquity of Kepler planetary systems and relate that to the multiplicity of planetary system. One of major components to reach that goal is to measure rotational period of stellar accurately. This seed paper is intended to constrain this value with the Kepler photometry data via Gaussian Process(GP) 

\end{abstract}

%% Keywords should appear after the \end{abstract} command. 
%% See the online documentation for the full list of available subject
%% keywords and the rules for their use.
\keywords{methods: data analysis ---
methods: statistical}

%% From the front matter, we move on to the body of the paper.
%% Sections are demarcated by \section and \subsection, respectively.
%% Observe the use of the LaTeX \label
%% command after the \subsection to give a symbolic KEY to the
%% subsection for cross-referencing in a \ref command.
%% You can use LaTeX's \ref and \label commands to keep track of
%% cross-references to sections, equations, tables, and figures.
%% That way, if you change the order of any elements, LaTeX will
%% automatically renumber them.
%%
%% We recommend that authors also use the natbib \citep
%% and \citet commands to identify citations.  The citations are
%% tied to the reference list via symbolic KEYs. The KEY corresponds
%% to the KEY in the \bibitem in the reference list below. 

\section{Introduction} \label{sec:intro}
\section{Data Acquisition and Reduction}
In order to test if our GP process inference can obtain consistent result as Ruth et al. (2017), we choose 756 Kepler Object of Interest(KOI) host stars, which had ACF calculation results in McQuillan et al. (2013a). 

We use the Pre-search Data Conditioning flux (pdcsap\_flux entry in Kepler light curve fits file) with median normalizing and unit-subtracting. Then we use periodic windows to mask out all known planet signals in the light curve. The length of the window is adopted as 2 times of transit duration, which are from NASA Exoplanet Archive. We also downsample the light curves using median box car with size of 13 to accelerate the calculation of likelihood when sampling.

We use \textbf{exoplanet} (Foreman-Mackey 2018), which is a toolkit for probabilistic modeling of transit and/or radial velocity observations of exoplanets and other astronomical time series using PyMC3. This python package includes a lot of  efficiently calculating their gradients so that they can be used with gradient-based inference methods like Hamiltonian Monte Carlo, No U-Turns Sampling, and variational inference. Therefore, this will improve out performance of inference.

The GP kernel we choose here is rotation kernel implemented in \textbf{exoplanet} package. This kernel is designed to sum two SHO term kernels to model the stellar activity (Foreman-Mackey et al. 2017). The power spectral density(PSD) of SHO term is:
\begin{equation}
    S(\omega) = \sqrt{\frac{2}{\pi}} \frac{S_{0}\omega^{4}}{(\omega^{2}-\omega_{0}^{2})^2+\omega_{0}^{2}\omega^{2}/Q^{2}}
\end{equation}
$S_{0}$, $Q$, $\omega_{0}$ are hyperparameters. For rotation term, 
\begin{eqnarray}
    \left\{\begin{array}{ll}
    Q_{1} = 0.5 +Q_{0} + deltaQ, \omega_{1} = \frac{4\pi Q_{1}}{P\sqrt{4Q_{1}^{2}-1}}, S_{1} = \frac{A}{\omega_{1}Q_{1}} \\
    Q_{2} = 0.5 +Q_{0}, \omega_{2} = \frac{8\pi Q_{2}}{P\sqrt{4Q_{2}^{2}-1}}, S_{2} = m \frac{A}{\omega_{2}Q_{2}}
\end{array}\right.
\end{eqnarray}
$P$ is period, $Q_{0}$ and $deltaQ$ are hyper-parameters about modes of oscillation frequency, $A$ is the amplitude of the variability and $m$ is the fractional amplitude of the secondary mode compared to the primary which should probably always be $0<m<1$

For non-period hyperparamters $Q_{0}$ and $deltaQ$ we use flat distribution as their prior input while for mix value we take uniform distribution $\mathcal{U} \sim (0,\,1)$. For the prior of rotation period, we conduct lomb-scargle analysis of every light curve with window range from 1 day to 100 days and sample rate is 50$/$peak. Then we use this value as the mean value of Gaussian distribution $\mathcal{N} \sim (ln(p),\,\sigma^{2})$ with $\sigma=5$. The prior distribution for the amplitude $A$ is adopted with $\mathcal{N} \sim (ln(var(lc)),\,\sigma^{2})$ where $var(lc)$ is the variance of the light curve, with $\sigma=5$ as well.

To sample the posterior in a way which is accurate and efficient enough, we use PyMC3 sampler. PyMC3 is a flexible and high-performance model building language and inference engine that scales well to problems with a large number of parameters. PyMC3 also implements No-U-Turn sampling (NUTS) that can eliminates setting of steps for Hamiltonian Monte Carlo. Therefore, NUTS can thus be used with no hand-tuning at all. We firstly use 200 tuning steps to run to learn the step size after tuning the mass matrix. We then use 2000 tuning iterations with target acceptance probability equals 0.9. The higher the latter value is , the smaller the step size is. Setting this to higher values like 0.9 can help with sampling from difficult posteriors. Finally we draw 2000 samples in each of 4 chains after discarding those samples in tuning steps. The fitting process for a single light curve take half an hour at least to 3 or 4 hours at most, depending on the behavior of the target covariance matrix.

\section{Results and Analysis}
After sampling all the posterior probability of these 756 KOI sources, we use the following standard to select those converged sampling for rotational period: 1) the number of effective samples should be larger than 1500; 2) Gelman-Rubin value should match with: $|\hat{R}-1|<0.0015$. Besides, we also check the results by human eyes after doing these two selection. Eventually, we obtain 669 converged posteriors while 93 unconverged ones. Next we will conduct clustering analysis about fitting parameters to see there are bound of all hyperparamters in the case of convergence.
\subsection{Hyperparamters analysis}

\begin{figure}
\plottwo{lnQ0vslndeltaQ.png}{lnQ1vslnQ2.png}
\caption{Relationship between $Q_0$ and $deltaQ$ and relationship between $Q$ value for two SHO terms\label{fig:f1}}
\end{figure}

\begin{figure}
\plottwo{lnw1vslnw2.png}{lnS1vslnS2.png}
\caption{Relationship between two $\omega$ and $S$ values for two SHO terms\label{fig:f2}}
\end{figure}

\begin{figure}
\plotone{psd.png}
\caption{power spectral density (PSD) for the combination of means of all hyperparamters\label{fig:f3}}
\end{figure}

\begin{figure}
\plotone{comparison_periods.png}
\caption{A comparison of our GP measurements with those in McQuillan et al. (2013a). The data points are coloured by the range of variability measured by McQuillan et al. (2013a), same plot notations as Ruth et al. (2017)\label{fig:f4}}
\end{figure}

\begin{figure}
\plotone{comparison_periods_Teff.png}
\caption{Same comparison between the two measurements. But the data points are coloured by the range of stellar effective temperature $T_{eff}$\label{fig:f5}}
\end{figure}

Unfortunately, we can not find any potential clustering features which contribute to constraining of hyperparamters by distinguishing converged posterior spaces from unconverged ones. 

\subsection{Some fitting failures}
We display some cases of failing GP fitting, which are classfied into four categories by human check: 1) bump-into-a-wall, which means that the posterior distribution of rotational period is truncated at some value (fig.7a); 2) twp-peak, the case that the posterior distribution appear two Gaussian peak (fig.9a); 3)tattered, the posterior samples are distributed totally random without forming Gaussian or half Gaussian distribution (fig.11a); 4) extreme-local-peaks, this one corresponds the scenario that there exists one or two extremely significant local maximums which is nearby or overlaps a broad Gaussian distribution in the period posterior distribution (fig. 13a).

\begin{figure}
\plotone{lombscargle_KOI41.png}
\caption{The lombscargle analysis of KOI-41}
\end{figure}

\begin{figure}
\plottwo{KOI-41.png}{posterior_hyper_parameters_KOI41_boundp.png}
\caption{bump-into-a-wall case, KOI-41. The left is the origin fitting without bound of the prior of period, the right is set bound for the prior distribution of ln(period) from 0 to 4. We can see that adopting proper bound can eliminate this unconverged case}
\end{figure}

\begin{figure}
\plotone{lombscargle_KOI505.png}
\caption{The lombscargle analysis of KOI-505}
\end{figure}

\begin{figure}
\plottwo{KOI-505.png}{posterior_hyper_parameters_KOI505_boundp.png}
\caption{two-peak case, KOI-505. Same plot caption as fig.7}
\end{figure}

\begin{figure}
\plotone{lombscargle_KOI387.png}
\caption{The lombscargle analysis of KOI-387}
\end{figure}

\begin{figure}
\plottwo{KOI-387.png}{posterior_hyper_parameters_KOI387_boundp.png}
\caption{tattered case, KOI-387. Same plot caption as fig.7}
\end{figure}

\begin{figure}
\plotone{lombscargle_KOI639.png}
\caption{The lombscargle analysis of KOI-639}
\end{figure}

\begin{figure}
\plottwo{KOI-639.png}{posterior_hyper_parameters_KOI639_boundp.png}
\caption{extreme-local-peak case, KOI-639. Same plot caption as fig.7}
\end{figure}

We found that after setting bound for the prior normal distribution of period, we can eliminate the first unconverged scenarios. However, for the the left situations we still don't have good solutions to import their fitting performances.

%% The reference list follows the main body and any appendices.
%% Use LaTeX's thebibliography environment to mark up your reference list.
%% Note \begin{thebibliography} is followed by an empty set of
%% curly braces.  If you forget this, LaTeX will generate the error
%% "Perhaps a missing \item?".
%%
%% thebibliography produces citations in the text using \bibitem-\cite
%% cross-referencing. Each reference is preceded by a
%% \bibitem command that defines in curly braces the KEY that corresponds
%% to the KEY in the \cite commands (see the first section above).
%% Make sure that you provide a unique KEY for every \bibitem or else the
%% paper will not LaTeX. The square brackets should contain
%% the citation text that LaTeX will insert in
%% place of the \cite commands.

%% We have used macros to produce journal name abbreviations.
%% \aastex provides a number of these for the more frequently-cited journals.
%% See the Author Guide for a list of them.

%% Note that the style of the \bibitem labels (in []) is slightly
%% different from previous examples.  The natbib system solves a host
%% of citation expression problems, but it is necessary to clearly
%% delimit the year from the author name used in the citation.
%% See the natbib documentation for more details and options.

\begin{thebibliography}{}

\end{thebibliography}

%% This command is needed to show the entire author+affilation list when
%% the collaboration and author truncation commands are used.  It has to
%% go at the end of the manuscript.
%\allauthors

%% Include this line if you are using the \added, \replaced, \deleted
%% commands to see a summary list of all changes at the end of the article.
%\listofchanges

\end{document}

% End of file `sample62.tex'.
